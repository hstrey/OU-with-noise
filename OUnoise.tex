% ****** Start of file OUnoise.tex ******
%%
\documentclass[%
 reprint,
%superscriptaddress,
%groupedaddress,
%unsortedaddress,
%runinaddress,
%frontmatterverbose, 
%preprint,
%showpacs,preprintnumbers,
%nofootinbib,
%nobibnotes,
%bibnotes,
 amsmath,amssymb,
 aps,
%pra,
%prb,
%rmp,
%prstab,
%prstper,
%floatfix,
]{revtex4-1}

\usepackage{graphicx}% Include figure files
\usepackage{dcolumn}% Align table columns on decimal point
\usepackage{bm}% bold math
\usepackage{subcaption}
\usepackage{float}
%\usepackage{hyperref}% add hypertext capabilities
%\usepackage[mathlines]{lineno}% Enable numbering of text and display math
%\linenumbers\relax % Commence numbering lines

%\usepackage[showframe,%Uncomment any one of the following lines to test 
%%scale=0.7, marginratio={1:1, 2:3}, ignoreall,% default settings
%%text={7in,10in},centering,
%%margin=1.5in,
%%total={6.5in,8.75in}, top=1.2in, left=0.9in, includefoot,
%%height=10in,a5paper,hmargin={3cm,0.8in},
%]{geometry}
\DeclareMathOperator\erf{erf}
\DeclareMathOperator\erfc{erfc}
\begin{document}

\preprint{APS/123-QED}

\title{Parameter estimation from an Ornstein-Uhlenbeck process with measurement noise}

\author{Helmut H. Strey}
 \affiliation{Biomedical Engineering Department and Laufer Center for Physical and Quantitative Biology, Stony Brook University, Stony Brook NY 11794-5281.}%Lines break automatically or can be forced with \\

\date{\today}% It is always \today, today,
             %  but any date may be explicitly specified

\begin{abstract}
\begin{description}
\item[PACS numbers]
May be entered using the \verb+\pacs{#1}+ command.
\end{description}
\end{abstract}

\pacs{Valid PACS appear here}% PACS, the Physics and Astronomy
                             % Classification Scheme.
%\keywords{Suggested keywords}%Use showkeys class option if keyword
                              %display desired
\maketitle

%\tableofcontents
\onecolumngrid
\subsection{Introduction}
\subsection{Probabilistic Description of Processes}
A probabilistic description of an overdamped Brownian particle in a harmonic potential (also called Ornstein-Uhlenbeck process) was first reported by Ornstein and Uhlenbeck in 1930 \cite{RN28}
\begin{equation}\label{OUp}
	x_{t+\Delta t} \sim \mathcal{N}(\mu=Bx_{t},\sigma^{2}=A(1-B^{2}))
\end{equation}
with $B(\Delta t) = \exp \left( { - \frac{\Delta t}{\tau}} \right)$ and $\mathcal{N}$ as a normal distribution.
This equation describes the conditional probability of finding a particle at time $t+\Delta t$ at $x_{t+\Delta t}$ given that it was at $x_{t}$ at time $t$.  The likelihood function for a specific time trace $\left\{x_i(i\Delta t)\right\}$ is:
\begin{equation}
	p\left( \left\{x_i(t_i)\right\} \left| B, A \right.\right) =
	\frac{1}{\sqrt {2 \pi A} }
	\exp \left( { - \frac{{x_1}^2}{2A}}\right)
	\frac{1}{{\sqrt {2\pi A(1-B^{2}(\Delta t))}^{(N-1)} }}
	\exp \left( { - \sum\limits_{i=1}^{N-1}\frac{{{{\left( {x_{i+1} - {x_i}B(\Delta t)} \right)}^2}}}{{2A(1-B^{2}(\Delta t))}}} \right)
\end{equation}
We now need to introduce measurement noise.  For right now, we will assume Gaussian noise that is added to each data point.  Such noise can be described as follows:
\begin{equation}
	y_{i} \sim \mathcal{N}(\mu=x_{i},\sigma=\sigma_{N})
\end{equation}
with likelihood function for $\{y_{i}\}$ given $\{x_{i}\}$ and $\sigma_{N}$:
\begin{equation}
	p\left( \left\{y_i(t_i)\right\} \left| \left\{x_i(t_i)\right\},\sigma_{N} \right.\right) =
	\frac{1}{{\sqrt {2\pi \sigma_{N}^{2}}^{N} }}
	\exp \left( { - \sum\limits_{i=1}^{N}\frac{{{{\left( {y_{i} - x_{i}B(\Delta t)} \right)}^2}}}{{2\sigma^{2}}}} \right)
\end{equation}

Since we are measuring $\{y_{i}\}$, $\{x_{i}\}$ are latent variables.  If we knew $\{x_{i}\}$ and $\{y_{i}\}$ the likelihood function would simply be.
\begin{equation}
	p\left( \{x_{i}\},\{y_{i}\}|A,B,\sigma_{N}\right) = 
	p\left( \left\{y_i(t_i)\right\} \left| \left\{x_i(t_i)\right\},\sigma_{N} \right.\right)
	p\left( \left\{x_i(t_i)\right\} \left| B, A \right.\right)
\end{equation}
But since we don't observe $\{x_{i}\}$, we need to marginalize over $\{x_{i}\}$:
\begin{equation}
p\left( \{y_{i}\}|A,B,\sigma_{N}\right) = \idotsint p\left( \{x_{i}\},\{y_{i}\}|A,B,\sigma_{N}\right) \,dx_1 \dots dx_N
\end{equation}
Using Bayes Theorem, we can express the conditional probabilities of $\{x_{i}\}$ in terms of the measured data $\{y_{i}\}$ and paramaters $\{A,B,\sigma_{N}\}$.
\begin{equation}
	p\left( \{x_{i}\}|\{y_{i}\},\{A,B,\sigma_{N}\}\right) =
	p\left( \{y_{i}\}|\{x_{i}\},\{A,B,\sigma_{N}\}\right)p\left(\{A,B,\sigma_{N}\}\right)
\end{equation}
\subsection{Implementation of the Estimation Maximization algorithm}
\begin{acknowledgments}
We wish to acknowledge funding by the 
\end{acknowledgments}

\end{document}
